\section{Preg'atirea 'si notarea laboratorului}

%Nota la acest laborator va fi format'a dup'a cum urmeaz'a:

%\subsection*{'Inainte de lucrarea practic'a: 40\%}
\subsection*{'Inainte de lucrarea practic'a: recomandare}

F'ar'a o pregatire prealabil'a a laboratorului, nu ve'ti avea timp s'a termina'ti lucrarea practic'a. De aceea v'a recomand'am urm'atoarele:

\begin{itemize}
\item[--] Citi'ti cu aten'tie sec'tiunile \thechapter.1, \thechapter.2, \thechapter.3 ale acestei lucr'ari.
\item[--] Rezolva'ti cele dou'a chestionare de antrenament de pe moodle p\^an'a 'in preziua laboratorului.
\item[--] Pentru bonus. Rezolva'ti exerci'tiile scriind r'aspunsuri pe foi, preg\^atind foi de calcul \textit{.xls}, circuite (\textit{.cir} sau \textit{.asc}). Prinde'ti toate aceste foi 'intr-un dosar de laborator.
\end{itemize}

%Studen'tii vor completa chestionarul preliminar ''Stiu deja'' (QUIZ 1) pe moodle, care con'tine no'tiuni introductive de teorie, poate fi rulat de oric\^ate ori 'si se inchide cu o zi 'inainte de laborator.
%
%Dup'a citirea conceptelor din acest document, studen'tii vor completa chestionarul preliminar ''Am 'in'teles'' (QUIZ 2) pe moodle, care verific'a 'in'telegerea conceptelir prezentate 'in lucrare, poate fi rulat de oric\^ate ori 'si se inchide cu o zi 'inainte de laborator.
%
%Studen'tii vor aduce un draft de referat de laborator (pe hartie 'si foi imprimate ale fi'selor de calcul/fi'sierelor Spice) care va con'tine rezolvarea exerci'tiilor din capitolul de Concepte 'si cel de Simul'ari.

\subsection*{'In timpul lucr'arii practice: 100\%}
Acest punctaj va fi acordat 'in urma experimentelor 'si complet'arii celui de-al treilea chestionar, 'in timpul laboratorului.

\subsection*{Dup'a lucrarea practic'a: pentru bonus}
Completa'ti un referat de laborator cu rezultatele experimentelor 'si concluzii personale. Documentul va fi scris de m\^an'a 'si va con'tine foi imprimate acolo unde este cazul. Referatul trebuie s'a aib'a un cuprins coerent, de exemplu:

\begin{enumerate}
\item Introducere
\item Rezolvarea exerci'tiilor 
\item Rezultate experimentale
\item Concluzii personale
\end{enumerate}